\chapter{Templates}


\begin{tempsection}
    % Examples 
    \figref{fig} is an example of a figure.
    \begin{figure}[htbp]
        \centerline{\includegraphics[width=0.3\columnwidth]{seal.png}}
        \caption{Example of a figure caption.}
        \label{fig}
    \end{figure}

    An example for multiple figures.
    \begin{figure}[htbp]
        \centering
        \begin{subfigure}{0.31\columnwidth}
            \centering
            \includegraphics[width=0.8\textwidth]{seal.png}
            \caption{Nice image1. \\ Another line.}
            \label{fig:multiple_figures_a}
        \end{subfigure}
        \begin{subfigure}{0.31\columnwidth}
            \centering
            \includegraphics[width=0.8\textwidth]{seal.png}
            \caption{Nice image 2. \\ Another line.}
            \label{fig:multiple_figures_b}
        \end{subfigure}
        \begin{subfigure}{0.31\columnwidth}
            \centering
            \includegraphics[width=0.8\textwidth]{seal.png}
            \caption{Nice image 3. \\ Another line.}
            \label{fig:multiple_figures_c}
        \end{subfigure}
        \caption{This is an example for multiple figures.}
        \label{fig:multiple_figures}
    \end{figure}

    Figure \ref{fig2} is an example of landscape figure.
    \begin{landscape}
        \begin{figure}[htbp]
            \centerline{\includegraphics[width=0.2\columnwidth]{seal.png}}
            \caption{Example of a figure caption.}
            \label{fig2}
        \end{figure}
    \end{landscape}

    This is an example table, \tbref{tb:this_is_table}.
    \begin{table}[htbp]
        \begin{center}
        \begin{threeparttable}
            \caption{Example table.\label{tb:this_is_table}}
            \centering
            \begin{tabular}{l c c c}
                \toprule
                \textbf{1st Column} & \textbf{2nd Column} & \textbf{3rd Column} & \textbf{4th Column} \\
                \midrule
                QWERTY\tnote{1}     &                     &                     &                     \\
                ASDFGH\tnote{2}     &                     &                     &                     \\
                \bottomrule
            \end{tabular}
            \begin{tablenotes}
                \item [1] Note 1.
                \item [2] Note 2.
            \end{tablenotes}
        \end{threeparttable}
        \end{center}
    \end{table}

    \begingroup
    \begin{table}
        \centering
        \caption{This is an example table.}\label{tbl:nicetablelesstable}
        \begin{tabular}{ll}
            What & is     \\
            \hline
            this & doing? \\
        \end{tabular}
    \end{table}
    \endgroup

    This is \coloredref{Assumption}{ass:ass1}.
    \begin{assumption}\label{ass:ass1}
        Test assumption
    \end{assumption}

    Please use `\textbackslash eqref' to refer an equation. Use `\textbackslash ref' to refer sections, figures or tables. For example, this is \secref{sec:1}, and this is \eqref{eq:1}.

    \begin{equation}
        E = mc^2
        \label{eq:1}
    \end{equation}

    \todomark{Testing todo}
\end{tempsection}