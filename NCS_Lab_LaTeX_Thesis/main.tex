% !TEX TS-program = xelatex
% !BIB program = bibtex
% !TEX encoding = UTF-8 Unicode

% --------------------------------------------------
% 國立臺灣大學電機工程學系網狀控制實驗室模板 - IEEE conference
% National Taiwan University (NTU) 
% Department of Electrical Engineering
% Thesis Template for Thesis
% https://github.com/NTU-NCS-lab/ThesisWritingTemplate

% 注意事項
% 使用此模板需自負檢查責任 http://www.lib.ntu.edu.tw/node/103
% --------------------------------------------------

% Comment this line to use the original style
\def\useNCSStyle{1}

% \documentclass[
%   twoside,
%   openright,
%   degree    = master,               % degree = master | doctor
%   language  = english,              % language = chinese | english
%   fontset   = template,             % fontset = default | template | system | overleaf
%   watermark = true,                 % watermark = true | false
%   doi       = true,                 % doi = true | false
% ]{ncs-thesis}
\documentclass{ncs-thesis}

\addbibresource{./back/references.bib}
% !TeX root = ./main.tex

% --------------------------------------------------
% 資訊設定(Information Configs)
% --------------------------------------------------

\ntusetup{
  university*   = {National Taiwan University},
  university    = {國立臺灣大學},
  college       = {電機資訊學院},
  college*      = {College of Electrical Engineering and Computer Science},
  institute     = {電機工程學系},
  institute*    = {Department of Electrical Engineering},
  title         = {國立臺灣大學碩博士畢業論文模版},
  title*        = {National Taiwan University (NTU) \\ Thesis/Dissertation Template in \LaTeX},
  author        = {朱雁丞},
  author*       = {Yen-Cheng Chu},
  ID            = {R10921008},
  advisor       = {連豊力},
  advisor*      = {Feng-Li Lian},
  date          = {2022-08-14},         % 若註解掉,則預設為當天
  oral-date     = {2022-08-14},         % 若註解掉,則預設為當天
  DOI           = {10.5566/NTU2018XXXXX},
  keywords      = {LaTeX, 中文, 論文, 模板},
  keywords*     = {LaTeX, CJK, Thesis, Template},
}

% --------------------------------------------------
% 加載套件(Include Packages)
% --------------------------------------------------

%\usepackage[sort&compress]{natbib}      % 參考文獻, use biblatex instead
\usepackage{amsmath, amsthm, amssymb}   % 數學環境
% \usepackage{ulem, CJKulem}              % 下劃線、雙下劃線與波浪紋效果
\usepackage{CJKulem}                    % 下劃線、雙下劃線與波浪紋效果
\usepackage{booktabs}                   % 改善表格設置
\usepackage{multirow}                   % 合併儲存格
\usepackage{diagbox}                    % 插入表格反斜線
\usepackage{array}                      % 調整表格高度
\usepackage{longtable}                  % 支援跨頁長表格
\usepackage{paralist}                   % 列表環境


\usepackage{lipsum}                     % 英文亂字
\usepackage{zhlipsum}                   % 中文亂字

\usepackage[normalem]{ulem}             % to strike the words

% --------------------------------------------------
% 套件設定(Packages Settings)
% --------------------------------------------------


\usepackage{pdfpages}                   % To insert the verification page

% 自定义引用格式:在内文引用中使用 et al. (如果有 3 个或更多作者)
\AtEveryCitekey{\ifnameundef{labelname}
  {\defcounter{maxnames}{1}}
  {\ifnum\value{listtotal}>2
    \defcounter{maxnames}{1}\defcounter{minnames}{1}\def\etalstring{et al.}\fi}}

% 自定义参考文献列表格式:显示所有作者
\AtEveryBibitem{\ifnameundef{labelname}
  {}
  {\defcounter{maxnames}{99}\defcounter{minnames}{1}}}


\begin{document}

% 封面與口試審定
% Cover and Verification Letter
\makecover                          % 論文封面(Cover)
\makeverification                   % 口試委員審定書(Verification Letter)

% 致謝與論文摘要
% Acknowledgement and Abstract
\input{front/acknowledgement}       % 致謝(Acknowledgement)
% !TeX root = ../main.tex

\begin{abstract}

    \begin{tempsection}
        中文摘要中文摘要中文摘要中文摘要中文摘要中文摘要中文摘要中文摘要中文摘要中文摘要中文摘要中文摘要中文摘要中文摘要中文摘要中文摘要中文摘要中文摘要中文摘要中文摘要中文摘要中文摘要中文摘要中文摘要中文摘要中文摘要中文摘要中文摘要中文摘要中文摘要中文摘要中文摘要中文摘要中文摘要中文摘要中文摘要中文摘要中文摘要中文摘要中文摘要中文摘要中文摘要中文摘要中文摘要中文摘要中文摘要中文摘要中文摘要中文摘要中文摘要中文摘要中文摘要中文摘要中文摘要中文摘要中文摘要中文摘要中文摘要中文摘要中文摘要中文摘要中文摘要中文摘要中文摘要中文摘要中文摘要中文摘要中文摘要中文摘要中文摘要中文摘要中文摘要中文摘要中文摘要中文摘要中文摘要中文摘要中文摘要中文摘要
    \end{tempsection}

\end{abstract}

\begin{abstract*}

    \begin{tempsection}
        Abstract
    \end{tempsection}

\end{abstract*}              % 摘要(Abstract)

% 生成目錄與符號列表
% Contents of Tables and Denotation
\maketableofcontents                % 目錄(Table of Contents)
\makelistoffigures                  % 圖目錄(List of Figures)
\makelistoftables                   % 表目錄(List of Tables)
\input{front/denotation}            % 符號列表(Denotation)



% Examples 
Figure \ref{fig} is an example of figure.
\begin{figure}[htbp]
    \centerline{\includegraphics[width=0.3\columnwidth]{seal.png}}
    \caption{Example of a figure caption.}
    \label{fig}
\end{figure}

Figure \ref{fig2} is an example of landscape figure.
\begin{landscape}
\begin{figure}[htbp]
    \centerline{\includegraphics[width=0.2\columnwidth]{seal.png}}
    \caption{Example of a figure caption.}
    \label{fig2}
\end{figure}
\end{landscape}

\begin{table}[htbp]
    \begin{center}
    \begin{threeparttable}
     \caption{Example table.\label{tab}}
     \centering
        \begin{tabular}{l c c c}
            \toprule
                \textbf{1st Column} & \textbf{2nd Column} & \textbf{3rd Column} & \textbf{4th Column} \\ 
            \midrule
                QWERTY\tnote{1}   &                     &                     &  \\
                ASDFGH\tnote{2}   &                     &                     &  \\ 
            \bottomrule
        \end{tabular}
        \begin{tablenotes}
            \item [1] Note 1. 
            \item [2] Note 2.
        \end{tablenotes}
    \end{threeparttable}
\end{center}
\end{table}

\begingroup
\begin{table}
    \centering
    \caption{This is an example table.}\label{tbl:nicetablelesstable}
    \begin{tabular}{ll}
        What & is     \\
        \hline
        this & doing? \\
    \end{tabular}
\end{table}
\endgroup

\begin{assumption}
    Test assumption
\end{assumption}

Please use `\textbackslash eqref' to refer an equation. Use `\textbackslash ref' to refer sections, figures or tables. For example, this is Section \ref{sec:1}, and this is equation \eqref{eq:1}.

\begin{equation}
    E = mc^2
    \label{eq:1}
\end{equation}

\todomark{Testing todo}

% 論文內容
% Contents of Thesis
\mainmatter
\input{contents/chapter01}
\input{contents/chapter02}
% !TeX root = ../main.tex

\chapter{中文測試}

\begin{tempsection}

\section{史記}

\zhlipsum[name=xiangyu]

\section{山海經}

\zhlipsum[name=nanshanjing]

\end{tempsection}
\input{contents/chapter04}

% 參考文獻
% References
\refmatter
\printbibliography

% 附錄
% Appendices
% !TeX root = ../main.tex

\appendix{A}{Introduction}

\begin{tempsection}
\section{Introduction}
\section{Further Introduction}
\end{tempsection}
% !TeX root = ../main.tex

\chapter{Introduction}
\begin{tempsection}
\section{Introduction}
\section{Further Introduction}
\end{tempsection}

\end{document}
